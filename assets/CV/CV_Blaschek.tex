\documentclass[11pt]{article}

\usepackage[paper=a4paper, total={150mm,220mm}]{geometry}
\usepackage{fontspec}
\setmainfont{Filosofia OT}
\usepackage[tracking=smallcaps,expansion=alltext,protrusion=true]{microtype}
\SetTracking[spacing={25*,166,}]{encoding=*,shape=sc}{50}
\usepackage[dvipsnames]{xcolor}
\usepackage{scrextend}
\usepackage{enumitem}
\usepackage{fancyhdr}
\usepackage{lastpage}
\pagestyle{fancy}
\fancyhf{}
\renewcommand{\headrulewidth}{0pt}
\cfoot{\thepage\ of \pageref{LastPage}}
\usepackage{hyperref}
\hypersetup{
	colorlinks=true,
	linkcolor=black,
	filecolor=magenta,
	urlcolor=black,
	pdfauthor={Leonard Blaschek},
	pdftitle={Curriculum Vitae - L. Blaschek}
}


\begin{document}
	\setlength\parindent{15pt}
\begin{center}
	\huge{\textsc{Leonard Blaschek}}
	\vspace*{1cm}
\end{center}

\noindent{\textsc{*01.03.1992} \hfill \raggedleft Svante Arrhenius väg 20A \\
	\href{mailto:leonard.blaschek@su.se}{leonard.blaschek@su.se} \hfill \raggedleft 114 18 Stockholm \\
	Phone: \textsc{+46 816 3793} \hfill \raggedleft Sweden \\
	\vspace{0.2cm}
	\centering\href{https://leonardblaschek.github.io}{\textcolor{ForestGreen}{leonardblaschek.github.io}
}

\vspace{1cm}
\raggedright

\Large \MakeUppercase{\textls[50]{education}} \normalsize } 
\vspace{0.2cm}

\textsc{\large{PhD, Plant Physiology}} \hfill \textsc{2017\textendash2021 \textnormal{(expected)}}
\begin{addmargin}[24pt]{0pt}
	\textit{Stockholms Universitet}, Sweden \\
	\textit{Project:} Functional and Genetic Analysis of Laccase Isoforms during Lignification \\
	\textit{Advisor:} Edouard Pesquet \\
	\textit{Co-Advisors:} Vincent Bulone, Jonas Gunnarsson
	\vspace{0.1cm} \\
	\small{My project aims to determine whether differences in lignin amount and composition on the cellular and organismal scale are due to distinct roles of laccase isoforms during lignification. Biochemical and genetic analyses of laccases from \textit{A. thaliana}, \textit{Zinnia violacea} and \textit{Populus} will be used to elucidate the basis of laccase non-redundancy as well as the evolutionary conservation of the roles of laccases in lignification.}
\end{addmargin}
\vspace{0.5cm}

\textsc{\large{Master of Science, Genetic and Molecular Plant Biology}} \hfill \textsc{2015\textendash2017}
\begin{addmargin}[24pt]{0pt}
	\textit{Uppsala Universitet}, Sweden \\
	\textit{Thesis:} Distinct Roles of Laccase Isoforms During Lignification in \textit{ A. thaliana}\\
	\textit{Advisor:} Edouard Pesquet
	\vspace{0.1cm} \\
	\small{In this thesis work, I provided evidence that laccase isoforms in \textit{A. thaliana} have distinct and non-redundant roles during lignification. Phenotypic analysis of \textit{laccase} loss-of-function mutants, \textit{in situ} activity assays and biochemical lignin characterisation showed that different laccase isoforms were active in a cell and substrate specific manner.}
\end{addmargin}
\vspace{0.5cm}

\textsc{\large{Bachelor of Science, Biology}} \hfill \textsc{2013\textendash2015}
\begin{addmargin}[24pt]{0pt}
	\textit{Ernst-Moritz-Arndt-Universität Greifswald}, Germany \\
	\textit{Thesis:} Plasma Membrane\textendash Bound Proteases in the Roots of \textit{H. vulgare} (grade: \textsc{1.0}) \\
	\textit{Advisor:} Gabriele Stöhr
	\vspace{0.1cm} \\
	\small{In my bachelor's thesis I investigated proteolytic activity in the plasma membrane of \textit{Hordeum vulgare} roots. Plasma membrane purification and SDS-PAGE analysis followed by zymographic and chromogenic activity assays provided evidence for the presence of an undescribed oligomeric membrane-bound aminopeptidase.}
\end{addmargin}
\vspace{1cm}

\noindent{\Large \MakeUppercase{\textls[50]{expertise}} \normalsize}
\vspace{0.2cm}

\textsc{\large{Physiology \& Cell Biology}} 
\begin{addmargin}[24pt]{0pt}
	Quantitative bright field and fluorescence microscopy, enzyme kinetics, agarose and polyacrylamide gel electrophoresis, cell suspension cultures, histology, UV-vis spectrophotometry and colour analysis 
\end{addmargin}
\vspace{0.2cm}

\pagebreak

\textsc{\large{Genetics}} 
\begin{addmargin}[24pt]{0pt}
	Cloning, transformation, crossing, segregation analysis, \textit{in silico} analyses
\end{addmargin}	
\vspace{0.2cm}

\textsc{\large{Coding \& Scripting}}
\begin{addmargin}[24pt]{0pt}
	R, LaTeX, ImageJ, git, HTML (basics), perl \& python (basics), Unix, Linux, Windows
\vspace{1cm}
\end{addmargin}

\noindent{\Large \MakeUppercase{\textls[50]{publications}} \normalsize}
	\begin{itemize}[label={},itemindent=-9pt,leftmargin=24pt]
	\item \textbf{Blaschek, L}, Champagne, A, Dimotakis, C, Décou, R, Kratzer, S, Hertzberg, M, Kajita, S, Pesquet, E (\textit{in preparation}). The Wiesner Test as a Quantitative Tool for Lignin Coniferaldehyde Determination \textit{in situ}.
	\end{itemize}
\vspace{1cm}

\noindent{\Large \MakeUppercase{\textls[50]{presentations}} \normalsize}
	\begin{itemize}[label={},itemindent=-9pt,leftmargin=24pt]
	\item \textbf{Blaschek, L} (2018). Determining the Spatial Distribution of Aldehyde Units in Lignin. \textit{2\textsuperscript{nd} Stockholm Cell Wall Meeting.}
	\end{itemize}
\vspace{1cm}

\noindent{\Large \MakeUppercase{\textls[50]{courses \& workshops}} \normalsize}
\begin{itemize}[label={},itemindent=-9pt,leftmargin=24pt]
	\item Advanced Imaging of Cells \textit{in vitro} and \textit{in vivo} (2018). \textit{Stockholm University} 
	\item Optical Clearing and Expansion Microscopy (2018). \textit{SciLifeLab, Stockholm} 	
	\item Advances in Enzyme Regulation (2018). \textit{Sveriges lantbruksuniversitet, Uppsala} 
\end{itemize}

\end{document}
