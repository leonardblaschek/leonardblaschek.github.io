\documentclass[11pt]{article}

\usepackage[paper=a4paper, total={150mm,220mm}]{geometry}
\usepackage{fontspec}
%\setmainfont{Filosofia OT}
\setmainfont{Bembo Std}
%\setmainfont{Minion Pro}
%\setmainfont{Adobe Caslon Pro}
\usepackage[tracking=smallcaps,expansion=alltext,protrusion=true]{microtype}
\SetTracking[spacing={25*,166,}]{encoding=*,shape=sc}{50}
\usepackage[dvipsnames]{xcolor}
\usepackage{scrextend}
\usepackage{enumitem}
\usepackage{textcomp}
\usepackage{fancyhdr}
\usepackage{lastpage}
\pagestyle{fancy}
\fancyhf{}
\renewcommand{\headrulewidth}{0pt}
\cfoot{\thepage\ of \pageref{LastPage}}
\usepackage{hyperref}
\hypersetup{
	colorlinks=true,
	linkcolor=black,
	filecolor=magenta,
	urlcolor=MidnightBlue,
	pdfauthor={Leonard Blaschek},
	pdftitle={Curriculum Vitae - L. Blaschek}
}


\begin{document}
	\setlength\parindent{15pt}
\begin{center}
	\huge{\textsc{Leonard Blaschek}}
	\vspace*{0.2cm}
\end{center}

\noindent{\href{mailto:leonard.blaschek@su.se}{leonard.blaschek@su.se} \hfill \raggedleft Svante Arrhenius väg 20A \\
	 \href{https://leonardblaschek.github.io}{leonardblaschek.github.io}\hfill \raggedleft 114 18 Stockholm, Sweden
	 %\textborn \space 01/03/1992 \hfill \raggedleft Sweden
	\vspace{0.2cm}


\vspace{0.75cm}
\raggedright

\large \MakeUppercase{\textls[50]{education}} \normalsize } 
\vspace{0.1cm}

\textsc{\large{PhD, Plant Physiology} \hfill \textsc{2017\textendash2022 \textnormal{(expected)}}}
\begin{addmargin}[24pt]{0pt}
	\textit{Stockholms Universitet}, Sweden \\
	\textit{Project:} Functional and Genetic Analysis of Laccase Isoforms during Lignification \\
	\textit{Advisor:} Dr. Edouard Pesquet \\
	\textit{Co-Advisors:} Prof. Vincent Bulone, Prof. Jonas Gunnarsson
	\vspace{0.1cm} \\
	\small{My project aims to determine whether differences in lignin amount and composition on the cellular and organismal scale are due to distinct roles of laccase paralogues during lignification. Biochemical and genetic analyses of laccases from \textit{A. thaliana}, \textit{Zinnia violacea} and \textit{Populus} will be used to elucidate the basis of laccase specificity as well as the evolutionary conservation of the roles of laccases in lignification.}
\end{addmargin}
\vspace{0.2cm}

\textsc{\large{Licentiate, Plant Physiology} \hfill \textsc{2017\textendash2020}}
\begin{addmargin}[24pt]{0pt}
	\textit{Stockholms Universitet}, Sweden \\
	\textit{Project:} Cellular Lignin Distribution Patterns and their Physiological Relevance \\
	\textit{Advisor:} Dr. Edouard Pesquet \\
	\textit{Co-Advisors:} Prof. Vincent Bulone, Prof. Jonas Gunnarsson\\
	\textit{Examination Committee:} Dr. András Gorzsás, Dr. Annelie Carlsbecker, Prof. Ulla Westermark
	\vspace{0.1cm} \\
	\small{In my licentiate thesis, I biochemically validated the Wiesner test and Raman microspectroscopy for reliable \textit{in situ} quantification of lignin composition, and subsequently used these techniques to characterise the distribution and genetic regulation of specific lignin subunits in different lignified cell types. Lastly we related this \textit{in situ} lignin composition data to single cell, tissue and whole plant morphology and bio-mechanics, showing that tracheary element functionality is dependent on specific compositions of \textit{post-mortem} deposited lignin.}
\end{addmargin}
\vspace{0.2cm}

\textsc{\large{Master of Science, Genetic and Molecular Plant Biology} \hfill \textsc{2015\textendash2017}}
\begin{addmargin}[24pt]{0pt}
	\textit{Uppsala Universitet}, Sweden \\
	\textit{Thesis:} Distinct Roles of Laccase Isoforms During Lignification in \textit{ A. thaliana}\\
	\textit{Advisor:} Dr. Edouard Pesquet
	\vspace{0.1cm} \\
	\small{In this thesis work, I provided evidence that laccase paralogs in \textit{A. thaliana} have distinct and non-redundant roles during lignification. Phenotypic analysis of \textit{laccase} loss-of-function mutants, \textit{in situ} activity assays and biochemical lignin characterisation showed that different laccase paralogs were active in a cell and substrate specific manner.}
\end{addmargin}
\vspace{0.2cm}

\textsc{\large{Bachelor of Science, Biology} \hfill \textsc{2013\textendash2015}}
\begin{addmargin}[24pt]{0pt}
	\textit{Ernst-Moritz-Arndt-Universität Greifswald}, Germany \\
	\textit{Thesis:} Plasma Membrane\textendash Bound Proteases in the Roots of \textit{H. vulgare}\\
	\textit{Advisor:} Prof. Christine Stöhr
	\vspace{0.1cm} \\
	\small{In my bachelor's thesis I investigated proteolytic activity in the plasma membrane of \textit{Hordeum vulgare} roots. Plasma membrane purification and SDS-PAGE analysis followed by zymographic and chromogenic activity assays provided evidence for the presence of an undescribed oligomeric membrane-bound aminopeptidase.}
\end{addmargin}
\vspace{0.5cm}

\newpage

\noindent{\large \MakeUppercase{\textls[50]{expertise}} \normalsize}

\textsc{\large{Practical}} 
\begin{addmargin}[24pt]{0pt}
	Quantitative bright field, fluorescence and vibrational microscopy, enzyme kinetics, image analysis, cell suspension cultures, histology, cloning, transformation, crossing
\end{addmargin}
\vspace{0.2cm}

\textsc{\large{Computational}} 
\begin{addmargin}[24pt]{0pt}
	R, LaTeX, ImageJ, git, HTML (basics), Python (basics), Linux, Windows
\end{addmargin}	
\vspace{0.5cm}

\noindent{\large \MakeUppercase{\textls[50]{courses \& workshops}} \normalsize}
\vspace{-0.175cm}
\begin{itemize}[label={},itemindent=-9pt,leftmargin=24pt]
	\itemsep-0.1cm
	\item Advanced Imaging of Cells \textit{in vitro} and \textit{in vivo} (2018). \textit{Stockholm University} 
	\item Optical Clearing and Expansion Microscopy (2018). \textit{SciLifeLab, Stockholm} 	
	\item Advances in Enzyme Regulation (2018). \textit{Swedish University of Agricultural Sciences, Uppsala} 
	\item Piecewise Structural Equation Modelling (2019). \textit{Stockholm University}
\end{itemize}
\vspace{0.3cm}

\noindent{\large \MakeUppercase{\textls[50]{publications}} \normalsize}
\vspace{-0.175cm}
\begin{itemize}[label={},itemindent=-9pt,leftmargin=24pt]
	\itemsep-0.1cm
	\item \textbf{Blaschek L}, Champagne A, Dimotakis C, Nuoendagula, Decou R, Hishiyama S, Kratzer S, Kajita S, Pesquet E (\textbf{2020}). Cellular and Genetic Regulation of Coniferaldehyde Incorporation in Lignin of Herbaceous and Woody Plants Using Quantitative Wiesner Staining. \textit{Front. Plant Sci.} 11:109. \href{https://doi.org/10.3389/fpls.2020.00109}{10.3389/fpls.2020.00109}
	\item \textbf{Blaschek L}\textsuperscript{$\dagger$}, Nuoendagula\textsuperscript{$\dagger$}, Bacsik Z, Kajita S, Pesquet E. (\textbf{2020}). Determining the Genetic Regulation and Coordination of Lignification in Stem Tissues of \textit{Arabidopsis} Using Semiquantitative Raman Microspectroscopy. \textit{ACS Sustain. Chem. Eng.}. \href{https://dx.doi.org/10.1021/acssuschemeng.0c00194}{10.1021/acssuschemeng.0c00194}
	\item \textbf{Blaschek L}, Pesquet E (\textit{in preparation}). Phenoloxidases: Functions, Structures and Evolution.
	\item Ménard D, Serk H, Gorzsás A, Jauneau A, Fukuda H, \textbf{Blaschek L}, Demura T, Goffner D, Pesquet E (\textit{in preparation}). The \textit{post-mortem} spatial restriction of lignification in protoxylem and metaxylem vessels in \textit{Zinnia elegans} is controlled by laccases and peroxidases.
	\item Ménard D, \textbf{Blaschek L}, Zhong C, Kriechbaum K, Lee CC, Nuoendagula, Kajita S, Mathew A, Pesquet E. (\textit{In preparation}). Lignin Ensures the Biomechanical Properties of Xylem Vessels under Tension.
\end{itemize}

\textsuperscript{$\dagger$}: contributed equally
\vspace{0.3cm}

\noindent{\large \MakeUppercase{\textls[50]{presentations}} \normalsize}
\vspace{-0.175cm}
\begin{itemize}[label={},itemindent=-9pt,leftmargin=24pt]
	\itemsep-0.1cm
	\item \textbf{Blaschek L} (2018). Determining the Spatial Distribution of Aldehyde Units in Lignin. \textit{2\textsuperscript{nd} Stockholm Cell Wall Meeting}, KTH Royal Institute of Technology, Stockholm.
	\item \textbf{Blaschek L} (2019). Spatial Distribution of Coniferaldehyde Lignin. \textit{28\textsuperscript{th} Congress of the Scandinavian Plant Physiology Society}, Umeå.
	\item \textbf{Blaschek L} (2019). The Structural Importance of Lignin in Xylem Vessels. \textit{3\textsuperscript{rd} Stockholm Cell Wall Meeting}, Stockholm University, Stockholm.
\end{itemize}
\vspace{0.3cm}

\noindent{\large \MakeUppercase{\textls[50]{grants \& scholarships}} \normalsize}
\vspace{-0.175cm}
\begin{itemize}[label={},itemindent=-9pt,leftmargin=24pt]
	\itemsep-0.1cm
	\item \textbf{Blaschek L}, Pesquet E (2018). Kungliga Vetenskapsakademien Scholarship BS2018--0061 for the sequencing of the \textit{Zinnia violacea} genome. 
	\item \textbf{Blaschek L} (2019). Travel grant of the Department of Ecology, Environment and Plant Sciences, Stockholm University to attend the 28\textsuperscript{th} Congress of the Scandinavian Plant Physiology Society.
\end{itemize}
\newpage
\end{document}

