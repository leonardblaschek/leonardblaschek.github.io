\documentclass[11pt]{article}

\usepackage[paper=a4paper, total={150mm,220mm}]{geometry}
\usepackage{fontspec}
%\setmainfont{Filosofia OT}
\setmainfont{Bembo Std}
%\setmainfont{Minion Pro}
%\setmainfont{Adobe Caslon Pro}
\usepackage[tracking=smallcaps,expansion=alltext,protrusion=true]{microtype}
\SetTracking[spacing={25*,166,}]{encoding=*,shape=sc}{50}
\usepackage{multicol}
\usepackage[dvipsnames]{xcolor}
\usepackage{scrextend}
\usepackage{enumitem}
\usepackage{textcomp}
\usepackage{fancyhdr}
\usepackage{lastpage}
\pagestyle{fancy}
\fancyhf{}
\renewcommand{\headrulewidth}{0pt}
\cfoot{\thepage\ of \pageref{LastPage}}
\usepackage{hyperref}
\hypersetup{
	colorlinks=true,
	linkcolor=black,
	filecolor=magenta,
	urlcolor=MidnightBlue,
	pdfauthor={Leonard Blaschek},
	pdftitle={Curriculum Vitae - L. Blaschek}
}

\newcommand*{\xdash}[1][3em]{\rule[0.5ex]{#1}{0.55pt}}

\begin{document}
	\setlength\parindent{15pt}
\begin{center}
	\huge{\textsc{Leonard Blaschek}}
	\vspace*{0.2cm}
\end{center}

\noindent{\href{mailto:leonard.blaschek@su.se}{leonard.blaschek@su.se} \hfill \raggedleft Svante Arrhenius väg 20A \\
	\href{https://leonardblaschek.github.io}{leonardblaschek.github.io}\hfill \raggedleft 114 18 Stockholm, Sweden
	%\textborn \space 01/03/1992 \hfill \raggedleft Sweden
	\vspace{0.2cm}
	
	
	\vspace{0.75cm}
	\raggedright
	
	\Large \textsc{Education} \normalsize } 
\vspace{0.1cm}

\textsc{\large{PhD, Plant Physiology} \hfill \textsc{2017\textendash 2022}}
\begin{addmargin}[24pt]{0pt}
	\textit{Stockholms Universitet}, Sweden \\
	\textit{Project:} Functional and Genetic Analysis of Laccase Isoforms during Lignification \\
	\textit{Supervisor:} Dr. Edouard Pesquet \\
	\textit{Co-Supervisors:} Prof. Vincent Bulone, Prof. Jonas Gunnarsson\\
	\textit{Examination committee:} Dr. Richard Sibout (opponent), Prof. Martin Lawoko, Dr. Anna Kärkönen, Prof. Igor Cesarino, Prof. Geoffrey Daniel, Dr. Mika Sipponen
	\vspace{0.1cm} \\
	%	\small{My project aims to determine whether differences in lignin amount and composition on the cellular and organismal scale are due to distinct roles of laccase paralogues during lignification. Biochemical and genetic analyses of laccases from \textit{A. thaliana}, \textit{Zinnia violacea} and \textit{Populus} will be used to elucidate the basis of laccase specificity as well as the evolutionary conservation of the roles of laccases in lignification.}
\end{addmargin}
\vspace{0.2cm}

\textsc{\large{Licentiate, Plant Physiology} \hfill \textsc{2017\textendash2020}}
\begin{addmargin}[24pt]{0pt}
	\textit{Stockholms Universitet}, Sweden \\
	\textit{Project:} Cellular Lignin Distribution Patterns and their Physiological Relevance \\
	\textit{Supervisor:} Dr. Edouard Pesquet \\
	\textit{Co-Supervisors:} Prof. Vincent Bulone, Prof. Jonas Gunnarsson\\
	\textit{Examination Committee:} Dr. András Gorzsás (opponent), Dr. Annelie Carlsbecker, Prof. Ulla Westermark
	\vspace{0.1cm} \\
	%	\small{In my licentiate thesis, I biochemically validated the Wiesner test and Raman microspectroscopy for reliable \textit{in situ} quantification of lignin composition, and subsequently used these techniques to characterise the distribution and genetic regulation of specific lignin subunits in different lignified cell types. Lastly we related this \textit{in situ} lignin composition data to single cell, tissue and whole plant morphology and bio-mechanics, showing that tracheary element functionality is dependent on specific compositions of \textit{post-mortem} deposited lignin.}
\end{addmargin}
\vspace{0.2cm}

\textsc{\large{Master of Science, Genetic and Molecular Plant Biology} \hfill \textsc{2015\textendash2017}}
\begin{addmargin}[24pt]{0pt}
	\textit{Uppsala Universitet}, Sweden \\
	\textit{Thesis:} Distinct Roles of Laccase Isoforms During Lignification in \textit{ A. thaliana}\\
	\textit{Supervisor:} Dr. Edouard Pesquet
	\vspace{0.1cm} \\
	%	\small{In this thesis work, I provided evidence that laccase paralogs in \textit{A. thaliana} have distinct and non-redundant roles during lignification. Phenotypic analysis of \textit{laccase} loss-of-function mutants, \textit{in situ} activity assays and biochemical lignin characterisation showed that different laccase paralogs were active in a non-redundant, cell-type specific manner.}
\end{addmargin}
\vspace{0.2cm}

\textsc{\large{Bachelor of Science, Biology} \hfill \textsc{2013\textendash2015}}
\begin{addmargin}[24pt]{0pt}
	\textit{Ernst-Moritz-Arndt-Universität Greifswald}, Germany \\
	\textit{Thesis:} Plasma Membrane\textendash Bound Proteases in the Roots of \textit{H. vulgare}\\
	\textit{Supervisor:} Prof. Christine Stöhr
	\vspace{0.1cm} \\
	%	\small{In my bachelor's thesis I investigated proteolytic activity in the plasma membrane of \textit{Hordeum vulgare} roots. Plasma membrane purification and SDS-PAGE analysis followed by zymographic and chromogenic activity assays provided evidence for the presence of an undescribed oligomeric membrane-bound aminopeptidase.}
\end{addmargin}
\vspace{0.5cm}

\noindent{\Large \textsc{Publications} \normalsize}

\hspace*{\fill} \xdash[6em] \large{\textsc{2021}} \xdash[6em] \hspace*{\fill} \normalsize

\vspace{-0.175cm}	
\begin{itemize}[label={},itemindent=-9pt,leftmargin=24pt]
	\itemsep-0.1cm
	\item \textbf{Blaschek L}, Pesquet E (2021). Phenoloxidases in Plants—How Structural Diversity Enables Functional Specificity. \textit{Front. Plant Sci.} 12, 2183.
	\href{https://doi.org/10.3389/fpls.2021.754601}{10.3389/fpls.2021.754601}
\end{itemize}


\hspace*{\fill} \xdash[6em] \large{\textsc{2020}} \xdash[6em] \hspace*{\fill} \normalsize

\vspace{-0.175cm}	
\begin{itemize}[label={},itemindent=-9pt,leftmargin=24pt]
	\itemsep-0.1cm
	\item Yamamoto M, \textbf{Blaschek L}, Subbotina E, Kajita S, Pesquet E (2020). Importance of Lignin Coniferaldehyde Residues for Plant Properties and Sustainable Uses. \textit{ChemSusChem} 13, 4400–4408.
	\href{https://doi.org/10.1002/cssc.202001242}{10.1002/cssc.202001242}
	\item \textbf{Blaschek L}\textsuperscript{$\dagger$}, Nuoendagula\textsuperscript{$\dagger$}, Bacsik Z, Kajita S, Pesquet E (2020). Determining the Genetic Regulation and Coordination of Lignification in Stem Tissues of \textit{Arabidopsis} Using Semiquantitative Raman Microspectroscopy. \textit{ACS Sustain. Chem. Eng.} 8, 4900--4909. \href{https://dx.doi.org/10.1021/acssuschemeng.0c00194}{10.1021/acssuschemeng.0c00194}
	\item \textbf{Blaschek L}, Champagne A, Dimotakis C, Nuoendagula, Decou R, Hishiyama S, Kratzer S, Kajita S, Pesquet E (2020). Cellular and Genetic Regulation of Coniferaldehyde Incorporation in Lignin of Herbaceous and Woody Plants Using Quantitative Wiesner Staining. \textit{Front. Plant Sci.} 11, 109. \href{https://doi.org/10.3389/fpls.2020.00109}{10.3389/fpls.2020.00109}
\end{itemize}

\hspace*{\fill} \xdash[6em] \large{\textsc{not yet peer-reviewed}} \xdash[6em] \hspace*{\fill}\normalsize

\vspace{-0.175cm}	
\begin{itemize}[label={},itemindent=-9pt,leftmargin=24pt]
	\itemsep-0.1cm
	\item \textbf{Blaschek L}, Murozuka E, Ménard D, Pesquet E (2022). Different combinations of laccase paralogs non-redundantly control the lignin amount and composition of specific cell types and cell wall layers in \textit{Arabidopsis}. \textit{bioRxiv.} \href{https://doi.org/10.1101/2022.05.04.490011}{10.1101/2022.05.04.490011}
	\item Ménard D\textsuperscript{$\dagger$}, \textbf{Blaschek L}\textsuperscript{$\dagger$}, Kriechbaum K, Lee CC, Serk H, Zhu C, Lyubartsev A, Nuoendagula, Bacsik Z, Bergström L, Mathew A, Kajita S, Pesquet E (2022). Plant biomechanics and resilience to environmental changes are controlled by specific lignin chemistries in each vascular cell type and morphotype. \textit{bioRxiv.} \href{https://doi.org/10.1101/2021.06.12.447240}{10.1101/2021.06.12.447240}
\end{itemize}

\textsuperscript{$\dagger$}: contributed equally
\vspace{0.5cm}

\noindent{\Large \textsc{Presentations} \normalsize}
\vspace{-0.175cm}
\begin{itemize}[label={},itemindent=-9pt,leftmargin=24pt]
	\itemsep-0.1cm
	\item \textbf{Blaschek L} (2021, selected talk). Laccase paralogs non-redundantly direct lignification. \textit{ASPB Plant Biology 2021,} online.
	\item \textbf{Blaschek L} (2021, selected talk). Specific and dynamic lignification at the cell-type level controls plant physiology and adaptability. \textit{SEB 2021 Annual Conference,} online. --- \href{https://leonardblaschek.github.io/talks.html}{link to recording}
	\item \textbf{Blaschek L} (2021, selected talk). Laccase paralogs non-redundantly direct lignification. \textit{SEB 2021 Annual Conference,} online.
	\item \textbf{Blaschek L} (2021, selected talk). Laccase paralogs non-redundantly direct lignification. \textit{7\textsuperscript{th} International Conference on Plant Cell Wall Biology,} online. --- \href{https://leonardblaschek.github.io/talks.html}{link to recording}
	\item \textbf{Blaschek L} (2019, selected talk). The structural importance of lignin in xylem vessels. \textit{3\textsuperscript{rd} Stockholm Cell Wall Meeting}, Stockholm University, Stockholm.
	\item \textbf{Blaschek L} (2019, selected talk). Spatial distribution of coniferaldehyde lignin. \textit{28\textsuperscript{th} Congress of the Scandinavian Plant Physiology Society}, Umeå.
	\item \textbf{Blaschek L} (2018, selected talk). Determining the spatial distribution of aldehyde units in lignin. \textit{2\textsuperscript{nd} Stockholm Cell Wall Meeting}, KTH Royal Institute of Technology, Stockholm.
\end{itemize}
\vspace{0.3cm}

\noindent{\Large \textsc{Grants,  scholarships \& awards} \normalsize}
\vspace{-0.175cm}
\begin{itemize}[label={},itemindent=-9pt,leftmargin=24pt]
	\itemsep-0.1cm
	\item \textbf{Blaschek L} (2021). Best student presentation award at the 7\textsuperscript{th} International Conference on Plant Cell Wall Biology.
	\item \textbf{Blaschek L} (2019). Travel grant of the Department of Ecology, Environment and Plant Sciences, Stockholm University to attend the 28\textsuperscript{th} Congress of the Scandinavian Plant Physiology Society.
	\item \textbf{Blaschek L}, Pesquet E (2018). Kungliga Vetenskapsakademien Scholarship BS2018--0061 for the sequencing of the \textit{Zinnia violacea} genome. 
\end{itemize}
\vspace{0.3cm}

\newpage

\noindent{\Large \textsc{Expertise} \normalsize}

\vspace{0.3cm}

\textsc{\large{Wet lab}} 
\vspace{-0.175cm}
\begin{addmargin}[24pt]{0pt}
	\begin{multicols}{2}
		\raggedright
		\begin{itemize}[itemindent=-9pt,leftmargin=24pt]
			\itemsep-0.1cm
			\item cloning (TA and Gateway)
			\item histology and histochemistry
			\item \textit{in vitro} plant systems (cell suspension cultures, seedlings, saplings)
			\item plant phenotyping, transformation  \& crossing (\textit{Arabidopsis, Populus, Zinnia})
			\item protein biochemistry (extraction, activity assays, SDS-PAGE)
			\item RT-qPCR
			\item quantitative bright field, fluorescence and vibrational microscopy
		\end{itemize}
	\end{multicols}
\end{addmargin}
\vspace{0.2cm}

\textsc{\large{Dry lab}} 
\vspace{-0.175cm}
\begin{addmargin}[24pt]{0pt}
	\begin{multicols}{2}
		\raggedright
		\begin{itemize}[itemindent=-9pt,leftmargin=24pt]
			\itemsep-0.1cm
			\item automated image analysis (Python, ImageJ)
			\item data analysis and visualisation (R, Python, bash)
			\item molecular phylogenetics
			\item protein homology modelling
			\item reproducible reporting (markdown, git)
			\item text processing (Office, LaTeX)
		\end{itemize}
	\end{multicols}
\end{addmargin}
\vspace{0.5cm}

\noindent{\Large \textsc{Courses \& Workshops} \normalsize}
\vspace{-0.175cm}
\begin{itemize}[label={},itemindent=-9pt,leftmargin=24pt]
	\itemsep-0.1cm
	\item Piecewise Structural Equation Modelling (2019). \textit{Stockholm University}
	\item Advanced Imaging of Cells \textit{in vitro} and \textit{in vivo} (2018). \textit{Stockholm University} 
	\item Optical Clearing and Expansion Microscopy (2018). \textit{SciLifeLab, Stockholm} 	
	\item Advances in Enzyme Regulation (2018). \textit{Swedish University of Agricultural Sciences, Uppsala} 
\end{itemize}
\vspace{0.3cm}

\noindent{\Large \textsc{Teaching} \normalsize}
\vspace{-0.175cm}
\begin{itemize}[label={},itemindent=-9pt,leftmargin=24pt]
	\itemsep-0.1cm
	\item Molecular plant--microbe interactions (MSc level). 2017--2020. Project design and supervision. \textit{Stockholm University} 	
	\item Green biotechnology (MSc level). 2018--2021. Project design and supervision. \textit{Stockholm University} 
\end{itemize}
\vspace{0.3cm}

\noindent{\Large \textsc{Service} \normalsize}
\vspace{-0.175cm}
\begin{itemize}[label={},itemindent=-9pt,leftmargin=24pt]
	\itemsep-0.1cm
	\item Member of the departmental equality group, \textit{Stockholm University}  \hfill 2019--2021
	\item Course representative in the department for evolutionary biology, \textit{Uppsala University} \hfill 2015
	\item Student representative in the board of the botanical institute, \textit{Universität Greifswald} \hfill 2014--2015
\end{itemize}

\end{document}

